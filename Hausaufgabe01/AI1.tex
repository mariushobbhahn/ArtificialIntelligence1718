\documentclass[a4paper,12pt]{scrartcl}
\usepackage[ngerman]{babel}
\usepackage{amsmath,amssymb,amsthm,amsfonts}
\usepackage{cancel}
\usepackage{graphicx}
\usepackage{ifthen}
\usepackage{tikz}
\usepackage[utf8]{inputenc} %Umlaute & Co
%\usepackage{fancyhdr}
\usepackage{enumerate}
\usepackage{changepage}
\usepackage{pgfplots}
\usepgfplotslibrary{fillbetween}
\usepackage{mathtools}
\usepackage{titlesec}
\usepackage{bm} %Fette Mathezeichen
\usepackage[onehalfspacing]{setspace}
\usepackage{scrpage2}
\usepackage{enumitem}
\DeclarePairedDelimiter{\ceil}{\lceil}{\rceil}
\DeclarePairedDelimiter{\floor}{\lfloor}{\rfloor}
\usepackage[normalem]{ulem}
\usetikzlibrary{trees,automata,arrows,shapes}
\pagestyle{empty}
\usepackage{marvosym}
\usepackage{gensymb}


\newcounter{aufgabe}
\def\tand{&}
\newcommand{\makeTableLine}[2][0]{%
	\setcounter{aufgabe}{1}%
	\whiledo{\value{aufgabe} < #1}%
	{%
		#2\tand\stepcounter{aufgabe}%
	}
}
\newcommand{\aufgTable}[1]{
	\def\spalten{\numexpr #1 + 1 \relax}
	\begin{tabular}{|*{\spalten}{p{1cm}|}}
		\makeTableLine[\spalten]{A\theaufgabe}$\Sigma$~~\\ \hline
		\rule{0pt}{15pt}\makeTableLine[\spalten]{}\\
	\end{tabular}
}
\newcounter{n}
\def\header#1#2#3#4#5#6#7{\pagestyle{empty}
	\begin{minipage}[t]{0.47\textwidth}
		\begin{flushleft}
			{\textbf{ #4}}\\
			#5\\
			Tutor: tba
		\end{flushleft}
	\end{minipage}
	\begin{minipage}[t]{0.5\textwidth}
		\begin{flushright}
			#6 \vspace{0.5cm}\\
			\aufgTable{#7}
		\end{flushright}
	\end{minipage}
	\vspace{1cm}
	\begin{center}
		{\Large\textbf{ Blatt #1}}
		
		{(Abgabe #3)}
	\end{center}}
	
\newcommand{\heading}[3]{\header{#1}{}{#2}{AI}{\textit{Marius Hobbhahn 4003731, \\ Marc Tomasek 3999493}
		}{WS 17/18}{#3}
		\vspace{1cm}}


\newcount\colveccount
\newcommand*\colvec[1]{
        \global\colveccount#1
        \begin{pmatrix}
        \colvecnext
}
\def\colvecnext#1{
        #1
        \global\advance\colveccount-1
        \ifnum\colveccount>0
                \\
                \expandafter\colvecnext
        \else
                \end{pmatrix}
        \fi
}



\newcommand{\R}{\mathbb{R}}
\newcommand{\N}{\mathbb{N}}
\renewcommand{\vec}[1]{\colvec{#1}}
\newcommand{\vecspace}[2]{\langle#1\rangle_{#2}}
\newcommand{\vecspaceR}[1]{\vecspace{#1}{\R}}

\renewcommand{\thesubsection}{(\alph{subsection})}

\newenvironment{subaufgaben}{\begin{enumerate}[label=\alph*)]}{\end{enumerate}}

\newcommand{\beh}{\textbf{Behauptung: \ }}
\newenvironment{beweis}{\textbf{Beweis: }}{\begin{center} $\Box$ \end{center}}
\newcommand{\set}[1]{\left\lbrace #1 \right\rbrace}
	
\begin{document}
	\titleformat{\section}[block]{}{}{0em}{\textbf{Aufgabe \thesection :} }{}
	\titleformat{\subsection}[block]{}{}{0em}{\thesubsection \ }{}
	\heading{1}{\today}{3} %Blatt No., Abgabe, Anzahl Aufgaben
	\pagestyle{scrheadings}
\section{}
	\begin{tabular}{l || l | l }
		environment & variable & justification \\ \hline
		Table Tennis & fully observable & I can see all relevant parameters at any given time \\
		& strategic & my choice of action is dependent on my opponent \\
		& sequential & the experience is continuous \\
		& dynamic & I need to react when my opponent shoots the ball back \\
		& continuous & no clearly defined percepts and actions \\
		& multi-agent & two-player game \\  \hline
		UNO & partially observable & I can't see what the next card is and what my opponents hand is \\
		& stochastic & the next card is uncertain \\
		& episodic & every play can be seen as an atomic action \\
		& static & I can take as much time as I want to deliberate \\
		& discrete & clearly define percepts and actions \\
		& multi-agent & there are opponents \\ \hline
		Archery & fully observable & I can see the bow, my hand and the target \\
		& deterministic & only my shot determines the outcome \\
		& episodic & the process of one shot can be defined as an episode \\
		& static & I can wait as long as I want before I shoot and the env does not change \\
		& discrete & one shot is a clearly defined episode \\
		& Single agent & my shot is independent of other players \\ \hline
		Autonomous  & partially observable & there are things the car might not be able to perceive \\
		& stochastic &  mostly strategic but there are random elements such as weather conditions\\
		& sequential & there are no reasonable atomic episodes \\
		& dynamic & environment changes over time while the care is deliberating \\
		& continuous & no limited number of clearly defined percepts and actions \\
		& multi-agent & there are other participants in traffic 
	\end{tabular}
\section{}
	\begin{tabular}{l || l | l }
		environment & variable & justification \\ \hline 
		Chess & multi-agent & two players \\
		& discrete & limited number of actions \\
		& fully observable & I can theoretically compute all states of the game \\
		& static & I can deliberate as long as I want \\ \hline
		one-armed bandit & Single-agent & I play alone \\
		& uncertain & I don't know the outcome of the machine (as a casual player) \\
		& discrete & limited number of actions and perceptions \\
		& static & I can wait as long as I want during rolls \\ \hline
		Car Driving & multi-agent & I share the traffic with other people \\
		& partially observable & I am unable to see everything in traffic \\
		& sequential & no atomic actions \\
		& continuous & no clearly defined actions and perceptions \\ \hline
		Puzzle & Single-agent & I play alone \\
		& fully observable & I know every piece and part  \\
		& static & environment unchanged while deliberating \\
		& discrete & limited number of actions, i.e. putting a piece somewhere \\ \hline 
		Skat & multi-agent & 3 players \\
		& discrete & limited number of actions  \\
		& partially observable & I don't know the hands of other players or the skat \\
		& static & technically I could wait for a long time to deliberate \\ \hline 
	\end{tabular}
\section{}
	
	 
\end{document}