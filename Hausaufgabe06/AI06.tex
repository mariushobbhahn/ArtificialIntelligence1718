\documentclass[12pt,a4paper]{scrartcl}
\usepackage[T1]{fontenc}
\usepackage[utf8x]{inputenc}
\usepackage{lscape}
\usepackage[ngerman]{babel}
\usepackage{fancyhdr}
\usepackage{amsmath}
\usepackage{amsthm}
\usepackage{amsfonts}
\usepackage{amssymb}
\usepackage{marvosym}
\usepackage{bbm}
\usepackage{mathtools}
\usepackage{array}
\usepackage{paralist}
\usepackage{tabularx}
\usepackage{caption}
\usepackage{listings}
\usepackage{multirow}
\usepackage[ruled]{algorithm}
%\usepackage{algorithmicx}
\usepackage{algpseudocode}
\usepackage{hyperref}
\usepackage{textcomp}
%\usepackage{ulsy}
\usepackage{array}
%Bäume, Automaten:
%\usepackage{qtree}
\usepackage{pgf}
\usepackage{tikz}
\usetikzlibrary{arrows,automata}
\usepackage{gensymb}
\usepackage{siunitx}
\usepackage{pgfplots}
%Seitenlayout
\usepackage{fancyhdr}
\usepackage{lastpage}
\pagestyle{fancy}
\renewcommand{\headrulewidth}{1pt}
\renewcommand{\footrulewidth}{0.4pt}
\fancyheadoffset{30pt}
\setlength{\headheight}{41pt}
\newcommand{\N}{\ensuremath{\mathbb{N}}}
\newcommand{\Z}{\ensuremath{\mathbb{Z}}}
\newcommand{\Q}{\ensuremath{\mathbb{Q}}}
\newcommand{\R}{\ensuremath{\mathbb{R}}}
\newcommand{\C}{\ensuremath{\mathbb{C}}}
\newcommand{\ggT}{\text{ggT}}
\newcommand{\abs}[1]{\ensuremath{ | #1 |}}
\newcommand{\norm}[1]{\ensuremath{ \lVert #1 \rVert}}
\newcommand{\qq}[1]{\glqq #1\grqq}
\newcommand{\todo}[1]{{\color{red}\textbf{TODO: #1}}}
\renewcommand{\matrix}[1]{\begin{pmatrix}#1\end{pmatrix}}
%Schicke Vektoren mit \vec{}
\newcount\colveccount
\newcommand*\colvec[1]{
	\global\colveccount#1
	\begin{pmatrix}
		\colvecnext
	}
	\def\colvecnext#1{
		#1
		\global\advance\colveccount-1
		\ifnum\colveccount>0
		\\
		\expandafter\colvecnext
		\else
	\end{pmatrix}
	\fi
}



\renewcommand{\labelenumi}{(\alph{enumi})}
%\renewcommand{\thesubsection}{Exercise \arabic{subsection}}
\lhead{Marius Hobbhahn\\Marc Tomasek}
\chead{{\Large Artificial Intelligence}\\ {\Large Übungsblatt 2}}
\rhead{\today}
\cfoot{{Seite \thepage} von \pageref*{LastPage}}
\begin{document}
	
\section{Simulated Annealing}

\section{LRTA*}
\subsection*{a)}
	\begin{tabular}{|l | l | l | l |}
		node & f-Value & explored & next \\ \hline
		a & b:4, e:2, f:7 & - & e \\
		e & b:4, c:8, d:8, g:8, f:7 & a & b \\
		b & c:7, d:8, g:8, f:7 & a,e & c \\
		c & d:8, g:8, f:7 & a,e,b & f \\
		f & d:8, g:8 & a,e,c,b & g \\
		g & finished & a,e,c,f,b & - 
	\end{tabular}\\\\
	Our Path is therefore $a \rightarrow b \rightarrow c \rightarrow d \rightarrow g$ with cost 8. Another path with the same cost would have been $a \rightarrow e \rightarrow g$ but we prioritize b over e due to alphabetical order.
\subsection*{b)}
	\begin{tabular}{l | l | l || l | l | l | l | l | l | l}
		a & s & s' & H[a] & H[b] & H[c] & H[d] & H[e] & H[f] & H[g] \\ \hline
		- & - & - & 3 & 3 & 3 & 1 & 1 & 1 & 0 \\
		- & - & a & 3 & 3 & 3 & 1 & 1 & 1 & 0 \\
		(a,e)  & a & e & 3 & 3 & 3 & 1 & 1 & 1 & 0 \\
		(e,a)  & e & a & 3 & 3 & 3 & 1 & 4 & 1 & 0 \\
		(a,b)  & a & b & 4 & 3 & 3 & 1 & 4 & 1 & 0 \\
		(b,a)  & b & a & 4 & 5 & 3 & 1 & 4 & 1 & 0 \\
		(a,e)  & a & e & 5 & 5 & 3 & 1 & 4 & 1 & 0 \\
		(e,a)  & e & a & 5 & 5 & 3 & 1 & 6 & 1 & 0 \\
		(a,b)  & a & b & 6 & 5 & 3 & 1 & 6 & 1 & 0 \\
		(b,c)  & b & c & 6 & 6 & 3 & 1 & 6 & 1 & 0 \\
		(c,d)  & c & d & 6 & 6 & 4 & 1 & 6 & 1 & 0 \\
		(d,g)  & d & g & 6 & 6 & 4 & 1 & 6 & 1 & 0 \\
	\end{tabular} \\
\subsection*{c)}
	we now use $H(n)$ instead of $h(n)$ as our heuristic. \\
	\begin{tabular}{l | l | l }
		n & h(n) & H(n) \\ \hline
		a & 3 & 6 \\
		b & 3 & 6 \\
		c & 3 & 4 \\
		d & 1 & 1 \\
		e & 1 & 6 \\
		f & 1 & 1 \\
		g & 0 & 0 
	\end{tabular}\\
	When now apply the normal A* algorithm we get:\\
	\begin{tabular}{|l | l | l | l |}
		node & f-Value & explored & next \\ \hline
		a & b:7, e:7, f:7 & -  & b \\
		b & c:8, e:7, f:7 & a & e \\
		e & d:8, c:8, f:7, g:8 & a,b & f \\
		f & d:8, c:8, g:8 & a,b,e  & c \\
		c & d:8, g:8 & a,b,e,f & d \\
		d & g:8 & a,b,e,f,c & g \\
		g & finished & a,b,e,f,c,d & - 
	\end{tabular}\\\\
	We can see that even though we might have a more accurate $H(n)$ it does not necessarily lead to a faster search. In this case we needed to expand more nodes. 
\end{document}
