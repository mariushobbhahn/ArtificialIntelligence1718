\documentclass[12pt,a4paper]{scrartcl}
\usepackage[T1]{fontenc}
\usepackage[utf8x]{inputenc}
\usepackage{lscape}
\usepackage[ngerman]{babel}
\usepackage{fancyhdr}
\usepackage{amsmath}
\usepackage{amsthm}
\usepackage{amsfonts}
\usepackage{amssymb}
\usepackage{bbm}
\usepackage{mathtools}
\usepackage{array}
\usepackage{paralist}
\usepackage{tabularx}
\usepackage{caption}
\usepackage{listings}
\usepackage{multirow}
\usepackage[ruled]{algorithm}
%\usepackage{algorithmicx}
\usepackage{algpseudocode}
\usepackage{hyperref}
\usepackage{textcomp}
%\usepackage{ulsy}
\usepackage{array}
%Bäume, Automaten:
%\usepackage{qtree}
\usepackage{pgf}
\usepackage{tikz}
\usetikzlibrary{arrows,automata}
\usepackage{gensymb}
\usepackage{siunitx}
%Seitenlayout
\usepackage{fancyhdr}
\usepackage{lastpage}
\pagestyle{fancy}
\renewcommand{\headrulewidth}{1pt}
\renewcommand{\footrulewidth}{0.4pt}
\fancyheadoffset{30pt}
\setlength{\headheight}{41pt}
\newcommand{\N}{\ensuremath{\mathbb{N}}}
\newcommand{\Z}{\ensuremath{\mathbb{Z}}}
\newcommand{\Q}{\ensuremath{\mathbb{Q}}}
\newcommand{\R}{\ensuremath{\mathbb{R}}}
\newcommand{\C}{\ensuremath{\mathbb{C}}}
\newcommand{\ggT}{\text{ggT}}
\newcommand{\abs}[1]{\ensuremath{ | #1 |}}
\newcommand{\norm}[1]{\ensuremath{ \lVert #1 \rVert}}
\newcommand{\qq}[1]{\glqq #1\grqq}
\newcommand{\todo}[1]{{\color{red}\textbf{TODO: #1}}}
\renewcommand{\matrix}[1]{\begin{pmatrix}#1\end{pmatrix}}
%Schicke Vektoren mit \vec{}
\newcount\colveccount
\newcommand*\colvec[1]{
	\global\colveccount#1
	\begin{pmatrix}
		\colvecnext
	}
	\def\colvecnext#1{
		#1
		\global\advance\colveccount-1
		\ifnum\colveccount>0
		\\
		\expandafter\colvecnext
		\else
	\end{pmatrix}
	\fi
}
\renewcommand{\labelenumi}{(\alph{enumi})}
%\renewcommand{\thesubsection}{Exercise \arabic{subsection}}
\lhead{Marius Hobbhahn\\Marc Tomasek}
\chead{{\Large Artificial Intelligence}\\ {\Large Übungsblatt 2}}
\rhead{\today}
\cfoot{{Seite \thepage} von \pageref*{LastPage}}
\begin{document}
	
\section{Greedy best-first search}
\subsection*{a)}
	\begin{tabular}{l | l | l   }
		node & options and its heuristic & option with best heuristic \\ \hline
		a & b: 17, d: 12 & d: 12 \\
		d & g: 10 ,a: 17 & g: 10 \\
		g & e: 8, i: 4, g:10& i: 4 \\
		i & h: 5, l: 0, i: 4 & l: 0 \\
		l & done & 
	\end{tabular}\\ \\
	The path we get from our heuristic therefore is: $a \rightarrow d \rightarrow g \rightarrow i \rightarrow l$\\
	The optimal path is either: $a \rightarrow d \rightarrow g \rightarrow i \rightarrow l$ or $a \rightarrow d \rightarrow g \rightarrow i \rightarrow h \rightarrow j \rightarrow k \rightarrow l$ as can be found out by a test with dijkstra. \\
	since the path that was found out is equivalent to one of the optimal paths we found the optimal solution. 
\subsection*{b)}
	\begin{tabular}{l | l | l   }
		node & options and its heuristic & option with best heuristic \\ \hline
		c & b:14, e: 10 & e:10 \\
		e & g: 9, f: 5, c: 13 & f:5 \\
		f & k: 0, h: 3, e: 10 & k:0 \\
		k & done 		
	\end{tabular}\\ \\
	The path we get from our heuristic therefore is: $c \rightarrow e \rightarrow f \rightarrow k$\\
	This is not the optimal path since the path $c \rightarrow e \rightarrow f \rightarrow h \rightarrow j \rightarrow k$ is shorter (since $14 < 19$). We did not find the optimal solution.
\subsection*{c)}
	\begin{tabular}{l | l | l   }
		node & options and its heuristic & option with best heuristic \\ \hline
		a & b: 13, d: 3 & d: 3 \\
		d & g: 8, a: 14 & g: 8 \\
		g & e: 6, i: 4 & i: 4 \\
		i & h: 2, l: 1 & l: 1\\
		l & k: 1, i: 4 & k: 1 \\
		k & f: 4, j: 0 & j:0 \\
		j & done   		
	\end{tabular}\\ \\
	The path we get from our heuristic therefore is: $a \rightarrow d \rightarrow g \rightarrow i \rightarrow l \rightarrow k \rightarrow j$. \\
	This is not the optimal path since $a \rightarrow d \rightarrow g \rightarrow i \rightarrow h \rightarrow j$ is shorter (since $14 < 20$). We did not find the optimal solution.
\section{Pathfinding with A*}
\subsection*{a)}
\subsection*{b)}
\subsection*{c)}
\subsection*{d)}
\section{A* in lisp}
	


\end{document}
