\documentclass[a4paper]{article}


\usepackage[utf8]{inputenc}
\usepackage[german]{babel}
\usepackage{amsmath}
\usepackage{amssymb}
\usepackage{fancyhdr}
\usepackage{color}
\usepackage{graphicx}
\usepackage{lastpage}
\usepackage{listings} 
\usepackage{tikz}
\usepackage{pdflscape}
\usetikzlibrary{trees}
\usepackage{subfigure}
\usepackage{float}
\usepackage{polynom}
\usepackage{hyperref}
\usepackage{tabularx}
\usepackage{forloop}
\usepackage{geometry}
\usepackage{listings}
\usepackage[]{algorithm2e}
\usepackage{fancybox}
\usepackage{tikz}
\usetikzlibrary{shapes}

\usepackage{algorithmic}
\usetikzlibrary{automata,arrows}

\usepackage{xparse}
\input kvmacros

%Größe der Ränder setzen
\geometry{a4paper,left=3cm, right=3cm, top=3cm, bottom=3cm}

%Kopf- und Fußzeile
\pagestyle {fancy}
\fancyhead[L]{}
\fancyhead[C]{Artificial Intelligence}
\fancyhead[R]{\today}

\fancyfoot[L]{}
\fancyfoot[C]{}
\fancyfoot[R]{Seite \thepage /\pageref*{LastPage} }


%Formatierung der Überschrift, hier nichts ändern
\def\header#1#2{
\begin{center}
{\Large\bf Übungsblatt #1} %Blatt eintragen

{(Abgabetermin #2)}
\end{center}
}

%Definition der Punktetabelle, hier nichts ändern
\newcounter{punktelistectr}
\newcounter{punkte}
\newcommand{\punkteliste}[2]{%
  \setcounter{punkte}{#2}%
  \addtocounter{punkte}{-#1}%
  \stepcounter{punkte}%<-- also punkte = m-n+1 = Anzahl Spalten[1]
  \begin{center}%
  \begin{tabularx}{\linewidth}[]{@{}*{\thepunkte}{>{\centering\arraybackslash} X|}@{}>{\centering\arraybackslash}X}
      \forloop{punktelistectr}{#1}{\value{punktelistectr} < #2 } %
      {%
        \thepunktelistectr & 
      } 
      #2 &  $\Sigma$ \\
      \hline
      \forloop{punktelistectr}{#1}{\value{punktelistectr} < #2 } %
      {%
        &
      } &\\ 
      \forloop{punktelistectr}{#1}{\value{punktelistectr} < #2 } %
      {%
        &
      } &\\ 
    \end{tabularx}
  \end{center}
}



\begin{document}

%Hier bitte Student 1 usw ersetzen
\begin{tabularx}{\linewidth}{m{0.2 \linewidth}X}
\begin{minipage}{\linewidth}%
%
% ----------------------- TODO ---------------------------
%Hier Namen eintragen
%
Marc Tomasek \\
Marius Hobbhahn \\
\end{minipage} & \begin{minipage}{\linewidth}%
%
% ----------------------- TODO ---------------------------
%Die zweite Zahl durch die Anzahl der Aufgaben ersetzen
%
%
\punkteliste{1}{1} %
%
\end{minipage}\\
\end{tabularx}



% ----------------------- TODO ---------------------------
%
%Hier Nummer und Datum aktualisieren
\header{Nr. 8}{20.12.2017}
%---------------------------------------------------------------------------
\section*{Aufgabe 1} %KCell
\subsection*{b} %Size-Stone-Tabellen
\begin{table}[h]
	\centering
	\caption{$\alpha\beta$-pruning times on top, Minimax-times below}
	\begin{tabular}{|c|r|r|r|r|r|}
		\hline
		\textbf{Stones \textbackslash Size} & \multicolumn{1}{c|}{\textbf{9}}                      & \multicolumn{1}{c|}{\textbf{10}}                     & \multicolumn{1}{c|}{\textbf{11}}                  & \multicolumn{1}{c|}{\textbf{12}}                    & \multicolumn{1}{c|}{\textbf{13}}                  \\ \hline
		\textbf{2}                          & \begin{tabular}[c]{@{}r@{}}38ms\\ 97ms\end{tabular}  & \begin{tabular}[c]{@{}r@{}}67ms\\ 444ms\end{tabular} & \begin{tabular}[c]{@{}r@{}}248ms\\ -\end{tabular} & \begin{tabular}[c]{@{}r@{}}444ms\\ -\end{tabular}   & \begin{tabular}[c]{@{}r@{}}-\\ -\end{tabular}     \\ \hline
		\textbf{3}                          & \begin{tabular}[c]{@{}r@{}}35ms\\ 108ms\end{tabular} & \begin{tabular}[c]{@{}r@{}}387ms\\ -\end{tabular}    & \begin{tabular}[c]{@{}r@{}}-\\ -\end{tabular}     & \begin{tabular}[c]{@{}r@{}}-\\ -\end{tabular}       & \begin{tabular}[c]{@{}r@{}}-\\ -\end{tabular}     \\ \hline
		\textbf{4}                          & \begin{tabular}[c]{@{}r@{}}3ms\\ 2ms\end{tabular}    & \begin{tabular}[c]{@{}r@{}}8ms\\ 10ms\end{tabular}   & \begin{tabular}[c]{@{}r@{}}209ms\\ -\end{tabular} & \begin{tabular}[c]{@{}r@{}}-\\ -\end{tabular}       & \begin{tabular}[c]{@{}r@{}}-\\ -\end{tabular}     \\ \hline
		\textbf{5}                          &                                                      &                                                      & \begin{tabular}[c]{@{}r@{}}3ms\\ 3ms\end{tabular} & \begin{tabular}[c]{@{}r@{}}14ms\\ 22ms\end{tabular} & \begin{tabular}[c]{@{}r@{}}688ms\\ -\end{tabular} \\ \hline
	\end{tabular}
\end{table}\noindent
Blank combinations are not possible. \\
Combinations marked with - take above $1$s to find.


\subsection*{c} %Size-Stone-Tabellen
The heuristic is very simple: It returns the average distance of your stones to the other side of the board. In a game where our heuristic is player1, it usually pushes its stones forward together. \\ 
Whereas if it is player2, it only pushes its first stone. It only moves the other stones if it has to. \\
After 25 Games each as player1 and player2 we come to the following statistics:

\begin{table}[h]
	\centering
	\caption{Heuristic is Player1}
	\begin{tabular}{|r|c|c|c|}
		\hline
		\textbf{Depth} & \multicolumn{1}{l|}{\textbf{Win}} & \multicolumn{1}{l|}{\textbf{Draw}} & \multicolumn{1}{l|}{\textbf{Loss}} \\ \hline
		\textbf{5}     & 9                                 & 11                                 & 5                                  \\ \hline
		\textbf{10}    & 7                                 & 15                                 & 3                                  \\ \hline
		\textbf{15}    & 8                                 & 13                                 & 4                                  \\ \hline
	\end{tabular}
	\caption{Heuristic is Player2}
	\begin{tabular}{|r|c|c|c|}
		\hline
		\textbf{Depth} & \multicolumn{1}{l|}{\textbf{Win}} & \multicolumn{1}{l|}{\textbf{Draw}} & \multicolumn{1}{l|}{\textbf{Loss}} \\ \hline
		\textbf{5}     & 16                                & 0                                  & 9                                  \\ \hline
		\textbf{10}    & 11                                & 1                                  & 13                                 \\ \hline
		\textbf{15}    & 14                                & 1                                  & 10                                 \\ \hline
	\end{tabular}
\end{table}
The search runs in under 5 seconds with a depth of upto $15$.


\end{document}