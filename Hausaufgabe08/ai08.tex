\documentclass[a4paper]{article}


\usepackage[utf8]{inputenc}
\usepackage[german]{babel}
\usepackage{amsmath}
\usepackage{amssymb}
\usepackage{fancyhdr}
\usepackage{color}
\usepackage{graphicx}
\usepackage{lastpage}
\usepackage{listings} 
\usepackage{tikz}
\usepackage{pdflscape}
\usetikzlibrary{trees}
\usepackage{subfigure}
\usepackage{float}
\usepackage{polynom}
\usepackage{hyperref}
\usepackage{tabularx}
\usepackage{forloop}
\usepackage{geometry}
\usepackage{listings}
\usepackage[]{algorithm2e}
\usepackage{fancybox}
\usepackage{tikz}
\usetikzlibrary{shapes}

\usepackage{algorithmic}
\usetikzlibrary{automata,arrows}

\usepackage{xparse}
\input kvmacros

%Größe der Ränder setzen
\geometry{a4paper,left=3cm, right=3cm, top=3cm, bottom=3cm}

%Kopf- und Fußzeile
\pagestyle {fancy}
\fancyhead[L]{}
\fancyhead[C]{Artificial Intelligence}
\fancyhead[R]{\today}

\fancyfoot[L]{}
\fancyfoot[C]{}
\fancyfoot[R]{Seite \thepage /\pageref*{LastPage} }


%Formatierung der Überschrift, hier nichts ändern
\def\header#1#2{
\begin{center}
{\Large\bf Übungsblatt #1} %Blatt eintragen

{(Abgabetermin #2)}
\end{center}
}

%Definition der Punktetabelle, hier nichts ändern
\newcounter{punktelistectr}
\newcounter{punkte}
\newcommand{\punkteliste}[2]{%
  \setcounter{punkte}{#2}%
  \addtocounter{punkte}{-#1}%
  \stepcounter{punkte}%<-- also punkte = m-n+1 = Anzahl Spalten[1]
  \begin{center}%
  \begin{tabularx}{\linewidth}[]{@{}*{\thepunkte}{>{\centering\arraybackslash} X|}@{}>{\centering\arraybackslash}X}
      \forloop{punktelistectr}{#1}{\value{punktelistectr} < #2 } %
      {%
        \thepunktelistectr & 
      } 
      #2 &  $\Sigma$ \\
      \hline
      \forloop{punktelistectr}{#1}{\value{punktelistectr} < #2 } %
      {%
        &
      } &\\ 
      \forloop{punktelistectr}{#1}{\value{punktelistectr} < #2 } %
      {%
        &
      } &\\ 
    \end{tabularx}
  \end{center}
}



\begin{document}

%Hier bitte Student 1 usw ersetzen
\begin{tabularx}{\linewidth}{m{0.2 \linewidth}X}
\begin{minipage}{\linewidth}%
%
% ----------------------- TODO ---------------------------
%Hier Namen eintragen
%
Marc Tomasek \\
Marius Hobbhahn \\
\end{minipage} & \begin{minipage}{\linewidth}%
%
% ----------------------- TODO ---------------------------
%Die zweite Zahl durch die Anzahl der Aufgaben ersetzen
%
%
\punkteliste{1}{3} %
%
\end{minipage}\\
\end{tabularx}



% ----------------------- TODO ---------------------------
%
%Hier Nummer und Datum aktualisieren
\header{Nr. 8}{20.12.2017}
%---------------------------------------------------------------------------

\section{CSP}
	\subsection*{a)}
	\begin{tabular}{l | l || l | l | l | l | l | l | l | l | l }
		line & assignment & A & B & C & D & E & F & G & H & I  \\ \hline
		0& init & rgb & rgb & rgb & rgb & rgb & rgb & rgb & rgb & rgb \\ \hline
		1 & A $\leftarrow$ r & r  & rgb & gb & rgb & rgb & rgb & rgb & rgb & rgb  \\ \hline
		2 & B $\leftarrow$ r & r & r & gb & gb & rgb  & rgb & rgb & rgb & rgb  \\ \hline
		3 & C $\leftarrow$ g & r & r & g & b & rgb & rb & rgb & rgb & rb \\ \hline
		4 & D $\leftarrow$ b & r & r & g & b & rg & rb & rg & rg & rb \\ \hline
		5 & E $\leftarrow$ r & r & r & g & b & r & rb & g & rg & rb \\ \hline
		6 & F $\leftarrow$ r & r & r & g & b & r & r & g & g & b  \\ \hline
		7 & G $\leftarrow$ g & r & r & g & b & r & r & g & \underline{g}* & b \\ \hline
		8 & F $\leftarrow$ b & r & r & g & b & r & b & g & r & r\\ \hline
		9 & G $\leftarrow$ g & r & r & g & b & r & b & g & r & r \\ \hline
		10 & H $\leftarrow$ r & r & r & g & b & r & b & g & r & \underline{r}** \\ \hline
		11 & E $\leftarrow$ g & r & r & g & b & g & rb & r & rg & rb \\ \hline
		12 & F $\leftarrow$ r & r & r & g & b & g & r & r & g & b \\ \hline
		13 & G $\leftarrow$ r & r & r & g & b & g & r & r & g & b \\ \hline
		14 & H $\leftarrow$ g & r & r & g & b & g & r & r & g & b \\ \hline
		15 & I $\leftarrow$ b & r & r & g & b & g & r & r & g & b \\ \hline
	\end{tabular}\\ \\
	* backtracks to line 5 before r was assigned to F and now assigns the other option b.\\
	** backtracks to line 4 before r was assigned to E and now assigns the other option g.\\
	This final coloring of the graph is consistent with our constraints.
	\subsection*{b)}
	\begin{tabular}{l | l || l | l | l | l | l | l | l | l | l }
		line & assignment & A & B & C & D & E & F & G & H & I  \\ \hline
		0 & init & rgb & rgb & rgb & rgb & rgb & rgb & rgb & rgb & rgb \\ \hline
		1 & A $\leftarrow$ r & r  & rgb & gb & rgb & rgb & rgb & rgb & rgb & rgb\\ \hline
		2 & B $\leftarrow$ r & r & r & gb & gb & rgb & rgb & rgb & rgb & rgb\\ \hline
		3 & C $\leftarrow$ g & r & r & g & b & rg & rb & rg & rg & rb \\ \hline
		4 & E $\leftarrow$ r & r & r & g & b & r & b & g & r & \underline{b}* \\ \hline
		5 & E $\leftarrow$ g & r & r & g & b & g & rb & r & g & rb \\  \hline
		6 & F $\leftarrow$ r & r & r & g & b & g & r & r & g & b \\ \hline
	\end{tabular}\\ \\
	We do not assign D in the 4th row  but skip it since it is already tested for consistency and would therefore be redundant.\\
	* in MAC we backtrack to the line before since we already know this leads to an inconsistent constellation. When we change E to g instead of r it works.	
	We are able to spot the inconsistency way earlier in the second example saving us much time. Therefore MAC has an advantage here.
\end{document}