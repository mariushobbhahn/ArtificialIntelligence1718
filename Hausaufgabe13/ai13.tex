\documentclass[a4paper]{article}


\usepackage[utf8]{inputenc}
\usepackage[german]{babel}
\usepackage{amsmath}
\usepackage{amssymb}
\usepackage{fancyhdr}
\usepackage{color}
\usepackage{graphicx}
\usepackage{lastpage}
\usepackage{enumitem}
\usepackage{listings} 
\usepackage{tikz}
\usepackage{pdflscape}
\usetikzlibrary{trees}
\usepackage{subfigure}
\usepackage{float}
\usepackage{polynom}
\usepackage{hyperref}
\usepackage{tabularx}
\usepackage{forloop}
\usepackage{geometry}
\usepackage{listings}
\usepackage[]{algorithm2e}
\usepackage{fancybox}
\usepackage{tikz}
\usetikzlibrary{shapes}

\usepackage{algorithmic}
\usetikzlibrary{automata,arrows}

\usepackage{xparse}
\input kvmacros

%Größe der Ränder setzen
\geometry{a4paper,left=3cm, right=3cm, top=3cm, bottom=3cm}

%Kopf- und Fußzeile
\pagestyle {fancy}
\fancyhead[L]{}
\fancyhead[C]{Artificial Intelligence}
\fancyhead[R]{\today}

\fancyfoot[L]{}
\fancyfoot[C]{}
\fancyfoot[R]{Seite \thepage /\pageref*{LastPage} }


%Formatierung der Überschrift, hier nichts ändern
\def\header#1#2{
\begin{center}
{\Large\bf Übungsblatt #1} %Blatt eintragen

{(Abgabetermin #2)}
\end{center}
}

%Definition der Punktetabelle, hier nichts ändern
\newcounter{punktelistectr}
\newcounter{punkte}
\newcommand{\punkteliste}[2]{%
  \setcounter{punkte}{#2}%
  \addtocounter{punkte}{-#1}%
  \stepcounter{punkte}%<-- also punkte = m-n+1 = Anzahl Spalten[1]
  \begin{center}%
  \begin{tabularx}{\linewidth}[]{@{}*{\thepunkte}{>{\centering\arraybackslash} X|}@{}>{\centering\arraybackslash}X}
      \forloop{punktelistectr}{#1}{\value{punktelistectr} < #2 } %
      {%
        \thepunktelistectr & 
      } 
      #2 &  $\Sigma$ \\
      \hline
      \forloop{punktelistectr}{#1}{\value{punktelistectr} < #2 } %
      {%
        &
      } &\\ 
      \forloop{punktelistectr}{#1}{\value{punktelistectr} < #2 } %
      {%
        &
      } &\\ 
    \end{tabularx}
  \end{center}
}



\begin{document}

%Hier bitte Student 1 usw ersetzen
\begin{tabularx}{\linewidth}{m{0.2 \linewidth}X}
\begin{minipage}{\linewidth}%
%
% ----------------------- TODO ---------------------------
%Hier Namen eintragen
%
Marc Tomasek \\
Marius Hobbhahn \\
\end{minipage} & \begin{minipage}{\linewidth}%
%
% ----------------------- TODO ---------------------------
%Die zweite Zahl durch die Anzahl der Aufgaben ersetzen
%
%
\punkteliste{1}{4} %
%
\end{minipage}\\
\end{tabularx}



% ----------------------- TODO ---------------------------
%
%Hier Nummer und Datum aktualisieren
\header{Nr. 13}{\today}
%---------------------------------------------------------------------------

\section{Unification}
\begin{enumerate}
	\item 
		$MGU = \{l/A, v/B, u/m\}$
	\item
	$MGU = \{m/l, l/S\}$
	\item
	$MGU = \{m/W(l,l), l/A, l/B\}$ \\
	This however is impossible, since we can't map the variable l to two different constants A and B.
	\item
	$MGU = \{m/A, v/B, l/P(l) \}$\\
	This however does not pass the occur check, since the variable itself occurs in the term it is mapped to.
	\item 
	$\{l/P(A,m), P \neq H  \}$\\
	P and H are not the same function, therefore a mapping is impossible.
\end{enumerate}
\section{Substitution and Skolemization}
\begin{enumerate}
	\item If we choose x as men, y as women and P(x,y) as x loves y. Then $\forall x \exists y P(x,y) \not \Rightarrow \exists q P(q,q)$, since there can be a woman or a man that does not love themself even if every man loves a woman. \\
	In formal termes we can make a existential instantiation such that $\forall x \exists P(x,y)$ is performed to $\forall x P(x, SK_0(x))$. Now we apply the universal instantiation with $x= \{A,B\}$, such that $SK_0(A) = B$ and $SK_0(B) = A$. Since $P(A,SK_0(A)) \land P(B,(SK_0(B)) \not \Rightarrow P(A,A) \lor P(B,B)$ the implication is not valid.
	\item 
	We start by inferring $\exists P(q,q)$ from $\forall x\exists y P(x,y)$. We follow with an existential instantiation: $\forall x \exists y P(x,y)$ is transferred to  $\forall x P(x, SK_0(x))$.\\
	With $x = \{x_1,x_2\}$ we can do universal instantiation such that $P(x_1, SK_0(x_1)) \land P(x_2, SK_0(x_2))$.\\
	With the broken engine we can infer $P(q_1,q_1) \land P(q_2,q_2)$.\\
	Because $q \in \{q_1,q_2\}$ we know $P(q_1,q_1) \land P(q_2,q_2)$ implies $\forall q P(q,q)$ and $\exists q P(q,q)$.
	\item 
	firstly we transform all mathematical to logical functions:
	\begin{itemize}
		\item $|x| \rightarrow Abs(x)$
		\item $ x > y \rightarrow gt(x,y)$
		\item[1] $\forall \epsilon gt(\epsilon,0) \exists n_0 \forall n gt(n,n_0) \Rightarrow gt(\epsilon, Abs(a_n))$
		\item[Ex. inst.] $\forall \epsilon gt(\epsilon,0) \forall n gt(n,SK_0(\epsilon)) \Rightarrow gt(\epsilon, Abs(a_n))$ 
		\item $\forall \epsilon \forall n gt(\epsilon,0) \land gt(n,SK_0(\epsilon)) \Rightarrow gt(\epsilon, Abs(a_n))$ 
		\item define $e \in \mathbb{R}$ and $m \in \mathbb{N}$
		\item $gt(e,0) \land gt(m, SK_0(e)) \Rightarrow GT(e, Abs(a_n))$
	\end{itemize}
	This can now be transformed to CNF:
	\begin{align*}
		gt(e,0) \land gt(m, SK_0(e)) \Rightarrow GT(e, Abs(a_n)) 
		&\equiv \neg (gt(e,0) \land gt(m, SK_0(e))) \lor GT(e, Abs(a_n))\\
		&\equiv \neg gt(e,0) \lor \neg gt(m, SK_0(e))) \lor GT(e, Abs(a_n))\\
	\end{align*}
	which is in CNF
\end{enumerate}

\section{Resolution in First-Order Logic}
We have a statement: $A(x) \land B(f(x), y)$ and a KB and want to find out through resolution whether the statement is constistent with the KB. Therefore we add the negation $\neg (A(x) \land B(f(x),y)) \equiv \neg A(x) \lor \neg B(f(x),y)$ to the KB:
\begin{itemize}
	\item[1] $B(f(x_1), x_2) \lor \neg A(x_2) \lor \neg C(x_1)$
	\item[2] $\neg D(y_1, y_2) \lor A(f(y_1))$
	\item[3] $C(f(G))$
	\item[4] $D(G,z_1)$
	\item[5] $\neg A(x) \lor \neg B(f(x),y)$
	\item[(2,4)] $A(f(G))$ with $\theta = \{y_1/G, y_2/z_1\}$
	\item[(5,(2,4))] $\neg B(f(f(G)), y_3)$ with $\theta = \{x_3/f(G)\}$
	\item[6 = (1,(5,(2,4)))] $\neg A(y_3) \lor \neg C(f(G))$ with $\theta = \{x_1/f(G), x_2/x_3 \}$
	\item[7 =(3,(1,(5,(2,4))))]$\neg A(y_3)$ with $\theta = identity$
	\item[6,7] $\square$
\end{itemize}


\end{document}