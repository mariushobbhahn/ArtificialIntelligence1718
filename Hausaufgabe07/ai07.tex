\documentclass[a4paper]{article}


\usepackage[utf8]{inputenc}
\usepackage[german]{babel}
\usepackage{amsmath}
\usepackage{amssymb}
\usepackage{fancyhdr}
\usepackage{color}
\usepackage{graphicx}
\usepackage{lastpage}
\usepackage{listings} 
\usepackage{tikz}
\usepackage{pdflscape}
\usetikzlibrary{trees}
\usepackage{subfigure}
\usepackage{float}
\usepackage{polynom}
\usepackage{hyperref}
\usepackage{tabularx}
\usepackage{forloop}
\usepackage{geometry}
\usepackage{listings}
\usepackage[]{algorithm2e}
\usepackage{fancybox}
\usepackage{tikz}
\usetikzlibrary{shapes}

\usepackage{algorithmic}
\usetikzlibrary{automata,arrows}

\usepackage{xparse}
\input kvmacros

%Größe der Ränder setzen
\geometry{a4paper,left=3cm, right=3cm, top=3cm, bottom=3cm}

%Kopf- und Fußzeile
\pagestyle {fancy}
\fancyhead[L]{}
\fancyhead[C]{Artificial Intelligence}
\fancyhead[R]{\today}

\fancyfoot[L]{}
\fancyfoot[C]{}
\fancyfoot[R]{Seite \thepage /\pageref*{LastPage} }


%Formatierung der Überschrift, hier nichts ändern
\def\header#1#2{
\begin{center}
{\Large\bf Übungsblatt #1} %Blatt eintragen

{(Abgabetermin #2)}
\end{center}
}

%Definition der Punktetabelle, hier nichts ändern
\newcounter{punktelistectr}
\newcounter{punkte}
\newcommand{\punkteliste}[2]{%
  \setcounter{punkte}{#2}%
  \addtocounter{punkte}{-#1}%
  \stepcounter{punkte}%<-- also punkte = m-n+1 = Anzahl Spalten[1]
  \begin{center}%
  \begin{tabularx}{\linewidth}[]{@{}*{\thepunkte}{>{\centering\arraybackslash} X|}@{}>{\centering\arraybackslash}X}
      \forloop{punktelistectr}{#1}{\value{punktelistectr} < #2 } %
      {%
        \thepunktelistectr & 
      } 
      #2 &  $\Sigma$ \\
      \hline
      \forloop{punktelistectr}{#1}{\value{punktelistectr} < #2 } %
      {%
        &
      } &\\ 
      \forloop{punktelistectr}{#1}{\value{punktelistectr} < #2 } %
      {%
        &
      } &\\ 
    \end{tabularx}
  \end{center}
}



\begin{document}

%Hier bitte Student 1 usw ersetzen
\begin{tabularx}{\linewidth}{m{0.2 \linewidth}X}
\begin{minipage}{\linewidth}%
%
% ----------------------- TODO ---------------------------
%Hier Namen eintragen
%
Marc Tomasek \\
Marius Hobbhahn \\
\end{minipage} & \begin{minipage}{\linewidth}%
%
% ----------------------- TODO ---------------------------
%Die zweite Zahl durch die Anzahl der Aufgaben ersetzen
%
%
\punkteliste{1}{3} %
%
\end{minipage}\\
\end{tabularx}



% ----------------------- TODO ---------------------------
%
%Hier Nummer und Datum aktualisieren
\header{Nr. 7}{13.12.2017}
%---------------------------------------------------------------------------
\section*{Aufgabe 1} %Simulated Annealing
\subsection*{a}
Let without loss of generality name one heap $a$ and the other one $b$. \\


\end{document}