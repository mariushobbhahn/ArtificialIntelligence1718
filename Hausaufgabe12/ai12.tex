\documentclass[a4paper]{article}


\usepackage[utf8]{inputenc}
\usepackage[german]{babel}
\usepackage{amsmath}
\usepackage{amssymb}
\usepackage{fancyhdr}
\usepackage{color}
\usepackage{graphicx}
\usepackage{lastpage}
\usepackage{listings} 
\usepackage{tikz}
\usepackage{pdflscape}
\usetikzlibrary{trees}
\usepackage{subfigure}
\usepackage{float}
\usepackage{polynom}
\usepackage{hyperref}
\usepackage{tabularx}
\usepackage{forloop}
\usepackage{geometry}
\usepackage{listings}
\usepackage[]{algorithm2e}
\usepackage{fancybox}
\usepackage{tikz}
\usetikzlibrary{shapes}

\usepackage{algorithmic}
\usetikzlibrary{automata,arrows}

\usepackage{xparse}
\input kvmacros

%Größe der Ränder setzen
\geometry{a4paper,left=3cm, right=3cm, top=3cm, bottom=3cm}

%Kopf- und Fußzeile
\pagestyle {fancy}
\fancyhead[L]{}
\fancyhead[C]{Artificial Intelligence}
\fancyhead[R]{\today}

\fancyfoot[L]{}
\fancyfoot[C]{}
\fancyfoot[R]{Seite \thepage /\pageref*{LastPage} }


%Formatierung der Überschrift, hier nichts ändern
\def\header#1#2{
\begin{center}
{\Large\bf Übungsblatt #1} %Blatt eintragen

{(Abgabetermin #2)}
\end{center}
}

%Definition der Punktetabelle, hier nichts ändern
\newcounter{punktelistectr}
\newcounter{punkte}
\newcommand{\punkteliste}[2]{%
  \setcounter{punkte}{#2}%
  \addtocounter{punkte}{-#1}%
  \stepcounter{punkte}%<-- also punkte = m-n+1 = Anzahl Spalten[1]
  \begin{center}%
  \begin{tabularx}{\linewidth}[]{@{}*{\thepunkte}{>{\centering\arraybackslash} X|}@{}>{\centering\arraybackslash}X}
      \forloop{punktelistectr}{#1}{\value{punktelistectr} < #2 } %
      {%
        \thepunktelistectr & 
      } 
      #2 &  $\Sigma$ \\
      \hline
      \forloop{punktelistectr}{#1}{\value{punktelistectr} < #2 } %
      {%
        &
      } &\\ 
      \forloop{punktelistectr}{#1}{\value{punktelistectr} < #2 } %
      {%
        &
      } &\\ 
    \end{tabularx}
  \end{center}
}



\begin{document}

%Hier bitte Student 1 usw ersetzen
\begin{tabularx}{\linewidth}{m{0.2 \linewidth}X}
\begin{minipage}{\linewidth}%
%
% ----------------------- TODO ---------------------------
%Hier Namen eintragen
%
Marc Tomasek \\
Marius Hobbhahn \\
\end{minipage} & \begin{minipage}{\linewidth}%
%
% ----------------------- TODO ---------------------------
%Die zweite Zahl durch die Anzahl der Aufgaben ersetzen
%
%
\punkteliste{1}{4} %
%
\end{minipage}\\
\end{tabularx}



% ----------------------- TODO ---------------------------
%
%Hier Nummer und Datum aktualisieren
\header{Nr. 12}{31.01.2017}
%---------------------------------------------------------------------------

\section{Resolution}
From the last assignment we know that there are 3 Variables: $Z,N,E$ and we have 5 statements that need to be fulfilled: $Z \lor N  \lor E, \neg Z \Rightarrow \neg N, E \Rightarrow N, E \Rightarrow \neg Z, Z  \Rightarrow (Z \land N) \lor (Z \land E)$\\
These can be transformed to CNF such that:
\begin{enumerate}
	\item $Z \lor N \lor E$
	\item $\neg Z \Rightarrow  \neg N \equiv Z \lor \neg N$
	\item $E \Rightarrow N \equiv \neg E \lor N$
	\item $E \Rightarrow \neg Z \equiv \neg E \lor \neg Z$
	\item $Z \Rightarrow (Z \lor N) \land (Z \lor E) \equiv \neg Z \lor Z \land (N \lor E) \equiv N \lor E$
\end{enumerate}
To make the proof by contradiction, we want to yield $KB \land \neg \alpha = false$, where $\alpha = Z \land N$ since this is the result from the last homework. $\neg \alpha$ is therefore $\neg(Z \land N) \equiv \neg Z \lor \neg N$(6).\\
Combination of 2 and 6 yields: $\neg N$(2,6) \\
Combination of 3 and 6 yields: $\neg Z \lor \neg E$(3,6) \\
Combination of 5 and 6 yields: $E \lor  \neg Z$(5,6) \\
Combination of (2,6) and 3 yields: $\neg E$ ((2,6),3)\\
Combination of (3,6) and (5,6) yields: $\neg Z$ ((3,6),(5,6))\\
Combination of (2,6), ((2,6),3), ((3,6), (5,6)) and 1 yields: $false$ \\

\section{More Resolution}
We want to prove $KB \land \neg \alpha = false$ with $\alpha = (C \land \neg D) \lor E$ from the knowledge base:
\begin{enumerate}
	\item $A \lor B$
	\item $B \lor C \lor E$
	\item $\neg B \lor C$
	\item $\neg B \lor \neg D$
	\item $\neg A \lor \neg D$
\end{enumerate}
Therefore $\neg \alpha = \neg((C \land \neg D) \lor E) \equiv \neg(C \land \neg D) \land \neg E \equiv (\neg C \lor D) \land \neg E$\\
This essentially means, we add two clauses: $(\neg C \lor D)$(6) and $\neg E$(7).\\
Combination of 1 and 4 yields: $A \lor \neg D$(1,4) \\
Combination of 2 and 7 yields: $(B \lor C)$ (2,7)\\
Combination of (2,7) and 3 yields: $C$ ((2,7),3)\\
Combination of (1,4) and 5 yields: $\neg D$((1,4),5)\\
Combination of ((2,7),3) and 6 yields: $D$ (((2,7),3), 6)\\
Combination of ((1,4),5) and (((2,7),3),6) yields: $false$ \\

\section{Clauses}
\subsection*{a)}
	%hier bin ich mir wirklich unsicher
	\begin{enumerate}
		\item is a correct representation, since it only allows the exam to be written, if and only if at least five assignments have been submitted.
		\item is also correct
		\item is incorrect, since it is still fulfilled if A is false
	\end{enumerate}
\subsection*{b)}
\begin{enumerate}
	\item $S \Rightarrow (E \Leftrightarrow A) \equiv \neg S \lor ((E \Rightarrow A) \land (A \Rightarrow E)) \equiv \neg S \lor ((\neg E \lor A) \land (\neg A \lor E)) \equiv (\neg A \lor E \lor \neg S) \land (A \lor  \neg E \lor \neg S)$ This is a horn clause, since in each clause only one literal is positive.
	\item $(S \land E) \Leftrightarrow A \equiv ((S \land E) \Rightarrow A) \land (A \Rightarrow (S \land E)) \equiv (\neg(S \land E) \lor A) \land (\neg A \lor (S \land E)) \equiv (\neg S \lor \neg E \lor A) \land (\neg A \lor S) \land (\neg A \lor E)$ This is a horn clause, since in each clause only one literal is positive.
	\item $S \Rightarrow (A \Rightarrow E) \equiv (\neg S \lor (\neg A \lor E)) \equiv (\neg S \lor \neg A \lor E)$. This is a horn clause, since only the E is positive.
\end{enumerate}
\section{First Order logic}
\begin{enumerate}
	\item $Occupation(James, student) \land Occupation(James, o)$
	\item $\neg Customer(James, Occupation(p, Architect))$
	\item $\exists occupation(p, architect): \forall customer(occupation(p1, scientiest),p)$
	\item $Boss(Occupation(p, Architect), Casey)$
	\item $\forall p \in  Physicists: Occupation(p, scientist)$
	\item $occupation(casey, physisist) \lor occupation(casey, architect)$
	\item $\neg \exists boss(occupation(p1, architect), occupation(p2, physicist))$
\end{enumerate}

\end{document}