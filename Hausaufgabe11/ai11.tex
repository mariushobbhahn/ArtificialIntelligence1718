\documentclass[a4paper]{article}


\usepackage[utf8]{inputenc}
\usepackage[german]{babel}
\usepackage{amsmath}
\usepackage{amssymb}
\usepackage{fancyhdr}
\usepackage{color}
\usepackage{graphicx}
\usepackage{lastpage}
\usepackage{listings} 
\usepackage{tikz}
\usepackage{pdflscape}
\usetikzlibrary{trees}
\usepackage{subfigure}
\usepackage{float}
\usepackage{polynom}
\usepackage{hyperref}
\usepackage{tabularx}
\usepackage{forloop}
\usepackage{geometry}
\usepackage{listings}
\usepackage[]{algorithm2e}
\usepackage{fancybox}
\usepackage{tikz}
\usetikzlibrary{shapes}

\usepackage{algorithmic}
\usetikzlibrary{automata,arrows}

\usepackage{xparse}
\input kvmacros

%Größe der Ränder setzen
\geometry{a4paper,left=3cm, right=3cm, top=3cm, bottom=3cm}

%Kopf- und Fußzeile
\pagestyle {fancy}
\fancyhead[L]{}
\fancyhead[C]{Artificial Intelligence}
\fancyhead[R]{\today}

\fancyfoot[L]{}
\fancyfoot[C]{}
\fancyfoot[R]{Seite \thepage /\pageref*{LastPage} }


%Formatierung der Überschrift, hier nichts ändern
\def\header#1#2{
\begin{center}
{\Large\bf Übungsblatt #1} %Blatt eintragen

{(Abgabetermin #2)}
\end{center}
}

%Definition der Punktetabelle, hier nichts ändern
\newcounter{punktelistectr}
\newcounter{punkte}
\newcommand{\punkteliste}[2]{%
  \setcounter{punkte}{#2}%
  \addtocounter{punkte}{-#1}%
  \stepcounter{punkte}%<-- also punkte = m-n+1 = Anzahl Spalten[1]
  \begin{center}%
  \begin{tabularx}{\linewidth}[]{@{}*{\thepunkte}{>{\centering\arraybackslash} X|}@{}>{\centering\arraybackslash}X}
      \forloop{punktelistectr}{#1}{\value{punktelistectr} < #2 } %
      {%
        \thepunktelistectr & 
      } 
      #2 &  $\Sigma$ \\
      \hline
      \forloop{punktelistectr}{#1}{\value{punktelistectr} < #2 } %
      {%
        &
      } &\\ 
      \forloop{punktelistectr}{#1}{\value{punktelistectr} < #2 } %
      {%
        &
      } &\\ 
    \end{tabularx}
  \end{center}
}



\begin{document}

%Hier bitte Student 1 usw ersetzen
\begin{tabularx}{\linewidth}{m{0.2 \linewidth}X}
\begin{minipage}{\linewidth}%
%
% ----------------------- TODO ---------------------------
%Hier Namen eintragen
%
Marc Tomasek \\
Marius Hobbhahn \\
\end{minipage} & \begin{minipage}{\linewidth}%
%
% ----------------------- TODO ---------------------------
%Die zweite Zahl durch die Anzahl der Aufgaben ersetzen
%
%
\punkteliste{1}{3} %
%
\end{minipage}\\
\end{tabularx}



% ----------------------- TODO ---------------------------
%
%Hier Nummer und Datum aktualisieren
\header{Nr. 11}{24.01.2017}
%---------------------------------------------------------------------------

\section{Propositional logic}
The entailment connection: $\vDash$ is true if and only if $True \vDash True$
\begin{itemize}
	\item[a)] Wrong by definition since $True \not \vDash False$
	\item[b)] There are cases where $(A \lor B)$ is true but $(A \Leftrightarrow B)$ is not, i.e. $A = 1, B = 0$
	\item[c)] 
	\begin{tabular}{c | c || c | c }
		A & B & $A \Leftrightarrow B$ & $A \land \neg B$ \\ \hline
		0 & 0 & 1 & 0 \\
		0 & 1 & 0 & 0 \\
		1 & 0 & 0 & 1 \\
		1 & 1 & 1 & 0 
	\end{tabular}\\
	There are cases in which $A \Leftrightarrow B$ is true but $A\land \neg B$ is not. It is therefore not entailed.
	\item[d)] Yes, since $(A \Rightarrow B) \equiv (\neg A \lor B)$, the entailment is therefore $True \vDash True$
	\item[e)] \begin{tabular}{c | c | c || c | c | c | c}
		A & B & C & $(B \Rightarrow A) \lor C$ & $(B \Rightarrow C) \land A$ & $(A \Rightarrow C) \land C$ & $e$ \\ \hline
		0 & 0 & 0 & 1 & 0 & 0 & 0\\
		0 & 0 & 1 & 1 & 0 & 1 & 1\\
		0 & 1 & 0 & 0 & 0 & 0 & 1\\
		0 & 1 & 1 & 1 & 0 & 1 & 1\\
		1 & 0 & 0 & 1 & 1 & 0 & 1\\
		1 & 0 & 1 & 1 & 1 & 1 & 1\\
		1 & 1 & 0 & 1 & 0 & 0 & 0\\
		1 & 1 & 1 & 1 & 1 & 1 & 1\\
	\end{tabular}\\
	Is satisfiable for e.g. $A = B = C = True$
	\item[f)] No, since there are cases in which $(\neg C \lor D \lor E)$ is true, but $(\neg D \lor \neg E)$ is false, i.e. $C = 0, D = 1, E = 1$. If now Both $A = 0$ and $B = 0$, then there is $True \vDash False$ which is wrong.
	\item[g)] $(A\land B) \land\neg (A \Rightarrow B) \equiv (A\land B) \land\neg (\neg A \lor B) \equiv (A\land B) \land (A \land  \neg B) \equiv A \land B \land \neg B$. This is a contradiction, meaning it is not satisfiable for any given input.
	\item[h)] \begin{tabular}{c | c | c || c | c }
		A & B & C & $(A \land B) \Rightarrow C$ & $(A \Rightarrow B) \lor (A \lor B)$ \\ \hline
		0 & 0 & 0 & 1 & 1 \\
		0 & 0 & 1 & 1 & 1 \\
		0 & 1 & 0 & 1 & 1 \\
		0 & 1 & 1 & 1 & 1 \\
		1 & 0 & 0 & 1 & 1 \\
		1 & 0 & 1 & 1 & 1 \\
		1 & 1 & 0 & 0 & 1 \\
		1 & 1 & 1 & 1 & 1 \\
	\end{tabular}\\
	In every case in which the first term is true, the second is as well making this a true entailment.
	\item[i)]
	\begin{tabular}{c | c | c || c | c | c | c}
		A & B & C & $(C \lor (A \land B))$ & $(A \Rightarrow B)$ & $(B \Rightarrow C)$ & $(\neg(A \Rightarrow B) \lor (B \Rightarrow C))$ \\ \hline
		0 & 0 & 0 & 0 & 1 & 1 & 1 \\
		0 & 0 & 1 & 1 & 1 & 1 & 1\\
		0 & 1 & 0 & 0 & 1 & 0 & 0\\
		0 & 1 & 1 & 1 & 1 & 1 & 1\\
		1 & 0 & 0 & 0 & 0 & 1 & 1\\
		1 & 0 & 1 & 1 & 0 & 1 & 1\\
		1 & 1 & 0 & 1 & 1 & 0 & 1\\
		1 & 1 & 1 & 1 & 1 & 1 & 1\\
	\end{tabular}\\
	In every case in which the first term is true, the second is as well making this a true entailment.
	\item[j)] \begin{tabular}{c | c || c | c | c }
		A & B & $(A \Leftrightarrow B)$ & $A \lor \neg B$ & j \\ \hline
		0 & 0 & 1 & 1 & 1 \\
		0 & 1 & 0 & 0 & 0 \\
		1 & 0 & 0 & 1 & 0 \\
		1 & 1 & 1 & 1 & 1 \\
	\end{tabular}\\
	Is satisfiable for e.g. $A = B = True$
\end{itemize}
\section{Normal forms}
\begin{itemize}
	\item[a)]
		\begin{align*}
			A \Rightarrow (B \lor C) &\equiv \neg A \lor (B \lor C) \\
			&\equiv (\neg A \lor B \lor C)
		\end{align*}
	\item[b)]
		\begin{align*}
			\neg A \Leftrightarrow (\neg B \land C) &\equiv (A \Rightarrow (\neg B \land C)) \land ((\neg B \land C) \Rightarrow A) \\
			&\equiv (\neg \neg A \lor (\neg B \land C)) \land (\neg(\neg B \land C) \lor A) \\
			&\equiv (A \lor (\neg B \land C)) \land ((B \land  \neg C) \lor A) \\
			&\equiv A \land ((\neg B \land C) \lor (B \land \neg C))\\
			&\equiv A \land (\neg B \lor B) \land (C \lor B) \land (\neg B \lor \neg C) \land (C \lor \neg C) \\
			&\equiv A \land (C \lor B)  \land (\neg B \lor \neg C)
		\end{align*}
	\item[c)]
		\begin{align*}
			(A \land B) \Rightarrow C &\equiv \neg(A \land B) \lor C \\
			&\equiv \neg A \lor \neg B \lor C \\
			&\equiv (\neg A \lor \neg B \lor C)
		\end{align*}
	\item[d)]
		\begin{align*}
			\neg A \lor (C \land B) \Rightarrow B &\equiv \neg(\neg A \lor (C \land B)) \lor B \\
			&\equiv (A \land \neg(C \land B)) \lor B \\
			&\equiv (A \land (\neg C \lor \neg B)) \lor B\\
			&\equiv (B \lor A) \land (B \lor (\neg C \lor \neg B)) \\
			&\equiv (B \lor A)  \land (B \lor \neg C \lor \neg B) \\
			&\equiv (B \lor A)  \land True \\
			&\equiv (B \lor A)
		\end{align*}
	\item[e)]
		\begin{align*}
			(\neg B \land (A \lor C)) \lor (A \land (B \lor C)) 
			&\equiv ((\neg B \land A) \lor (\neg B \land C)) \lor ((A \land B) \lor (A \land C)) \\
			&\equiv (\neg B \land A) \lor (\neg B \land C) \lor (A \land B) \lor (A \land C) \\
			&\equiv (\neg B \land A) \lor (A \land B) \lor (\neg B \land C) \lor  (A \land C) \\
			&\equiv (A \land (B \lor \neg B)) \lor (C \land (A \lor \neg B))\\
			&\equiv A \lor (C \land (A \lor \neg B)) \\
			&\equiv (A \lor C)  \land (A \lor (A \lor \neg B))\\
			&\equiv (A \lor C) \land  (A \lor A \lor \neg B)\\
			&\equiv (A \lor C) \land (A \lor \neg B)
		\end{align*}
\end{itemize}
\section{The hacking case}
	\subsection*{a)}
		\begin{enumerate}
			\item $Z \lor N \lor E$
			\item $Z  \Rightarrow (Z \land N) \lor (Z \land E)$
			\item $\neg Z \Rightarrow \neg N$
			\item $E \Rightarrow N$
			\item $E \Rightarrow \neg Z$
		\end{enumerate}
	\subsection*{b)}
	The additional assumption that we make here is that $1 \land 2 \land 3 \land 4  \land 5$. That implies that as soon as one statement is not fulfilled we can already exclude the row from the realm of possibilities.\\
	\begin{tabular}{c | c | c || c | c | c | c | c || c }
		Z & N & E & $Z \lor N  \lor E$ &  $\neg Z \Rightarrow \neg N$ & $E \Rightarrow N$ & $E \Rightarrow \neg Z$ & $Z  \Rightarrow (Z \land N) \lor (Z \land E)$ & All \\ \hline
		0 & 0 & 0 & 0 &  - & -  & - & - & -  \\
		0 & 0 & 1 & 1 & 1 & 0 & - & - & -  \\
		0 & 1 & 0 & 1 & 0 & - & - & - & -  \\
		0 & 1 & 1 & 1 & 0 & - & - & - & - \\
		1 & 0 & 0 & 1 & 1 & 1 & 1 & 0 & - \\
		1 & 0 & 1 & 1 & 1 & 0 & - & - & - \\
		1 & 1 & 0 & 1 & 1 & 1 & 1 & 1 & 1 \\
		1 & 1 & 1 & 1 & 1 & 1 & 0 & - & - \\
	\end{tabular}
	\\
	We can therefore conclude that Zack and Naomi are the hackers.
\end{document}